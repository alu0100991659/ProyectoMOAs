\chapter{Adaptive Sliding Windows}
\label{ch:adwin}

\BEGINOMIT
In this chapter, we present %propose to study and develop 
several optimal strategies 
for learning with data whose nature changes over time
%. This current work constitutes our contributions in this area of research,which can be divided in two main sections
 : an algorithm {\tt ADWIN} to detect change using an adaptive sliding window, and the combination of {\tt ADWIN} with Kalman filters.
\ENDOMIT

Dealing with data whose nature changes over time
is one of the core problems in data mining and machine learning.
%To mine or learn such data, one needs strategies for
%the following three tasks, at least: 1) detecting when
%change occurs 2) deciding which examples to keep and which ones
%to forget (or, more in general, keeping updated sufficient statistics),
%and 3) revising the current model(s) when significant  
%change has been detected.
In this chapter we present {\tt ADWIN}, an adaptive sliding window
algorithm, as an estimator with memory and change detector with the
main properties of optimality explained in section~\ref{sOptimal}. 
We study and develop also the combination of {\tt ADWIN} with Kalman filters.


\section{Introduction}
\label{Introduction}

\BEGINOMIT
Dealing with data whose nature changes over time
is one of the core problems in data mining and machine learning.
%To mine or learn such data, one needs strategies for
%the following three tasks, at least: 1) detecting when
%change occurs 2) deciding which examples to keep and which ones
%to forget (or, more in general, keeping updated sufficient statistics),
%and 3) revising the current model(s) when significant  
%change has been detected.
In this chapter we prosose {\tt ADWIN}, an adaptive sliding window
algorithm, as an estimator with memory and change detector with the
main properties of optimality explained in section~\ref{sOptimal}. 
\ENDOMIT

Most strategies in the literature use variations of the {sliding window} idea:
a window is maintained that keeps the most recently read examples,
and from which older examples are dropped according to some
set of rules. The contents of the window can be used for the
three tasks: 1) to detect change (e.g., by using some statistical
test on different subwindows), 2) obviously, to
obtain updated statistics from the recent examples,
and 3) to have data to rebuild or revise the model(s) after data has 
changed.

The simplest rule is to keep a window
of some fixed size, usually determined {\em a priori} by the user.
This can work well if information on the time-scale
of change is available, but this is rarely the case.
Normally, the user is caught in a tradeoff without solution:
choosing a small size (so that the window reflects accurately the current distribution)
and choosing a large size (so that many examples are available to work on, 
increasing accuracy in periods of stability).
A different strategy uses a {\em decay function}
to weight the importance of examples according to their
age (see e.g. \cite{CS03}):
the relative contribution of each data item 
 is scaled down by a factor that depends on %, and is non-decreasing with,
elapsed time.
 %strauss
In this case, the tradeoff shows up in the 
choice of a decay constant that should match the unknown rate of change.


Less often, it has been proposed to use windows of variable size.
In general, one tries to keep examples as long as possible, i.e., 
while not proven stale. This delivers
the users from having to guess {\em a priori} an unknown parameter such
as the time scale of change. However, most works along these lines 
that we know of (e.g., \cite{Gama,Klinkenberg,olin,WidmerKubat})
are heuristics and have no rigorous guarantees of performance. 
Some works in computational learning theory 
(e.g. \cite{bartlett00,helmbold94tracking,herbster95tracking}) 
describe strategies with rigorous performance
bounds, but to our knowledge they have never been tried
in real learning/mining contexts and often assume a known bound 
on the rate of change. 

We will present {\tt ADWIN}, a parameter-free adaptive size sliding
window, with theoretical garantees. We will
use Kalman filters at the last part of this Chapter, in order
to provide an adaptive weight for each item. % according to their age.



%\ENDOMIT

\input{ConTechnical}

\def\adwin{{\tt ADWIN} }

%%%%%%%%%%%%%%%%%%%%%%%%%%%%%%%%%%%%%%%%%%%%%%%%%%%%%%%

\section{{\tt K-ADWIN} = {\tt ADWIN} + Kalman Filtering}
\label{Skadwin}

%In the sequel, whenever we say {\tt ADWIN} we really mean its
%efficient implementation, {\tt ADWIN2}. 

%\subsubsection{The Kalman Filter}
%\label{Sskalman}

One of the most widely used Estimation algorithms is the Kalman filter. We give here a description
of its essentials; see \cite{welch} for a complete introduction.
%The Kalman filter is an optimal recursive data-processing algorithm that generates estimates of the variables 
%(or states) %of the system being controlled by processing all available measurements. 

The Kalman filter addresses the general problem of trying to estimate the state $x \in \Re^n$ 
of a discrete-time controlled process that is governed by the linear stochastic difference equation
$$x_t=Ax_{t-1} + B u_t + w_{t-1}$$
with a measurement $z \in \Re^m$ that is
$$Z_t = H x_t + v_t.$$
%
The random variables $w_t$ and $v_t$ represent the process and measurement noise
(respectively). They are assumed to be independent (of each other), white, and with
normal probability distributions
$$p(w) \sim N(0,Q) $$
$$p(v) \sim N(0,R). $$
%
In essence, the main function of the Kalman filter is to estimate the state vector 
using system sensors and measurement data  corrupted by noise.

The Kalman filter estimates a process by using a form of feedback control: the filter
estimates the process state at some time and then obtains feedback in the form of (noisy)
measurements. As such, the equations for the Kalman filter fall into two groups: time
update equations and measurement update equations. The time update equations are
responsible for projecting forward (in time) the current state and error covariance
estimates to obtain the a priori estimates for the next time step. 
$$x^-_t=A x_{t-1} + B u_t$$
$$P^-_t= AP_{t-1} A^T +Q$$
%
The measurement update equations are responsible for the feedback, i.e. for 
incorporating a new measurement into the a priori estimate to obtain an improved a posteriori estimate.
%
$$K_t=P^-_t H^T(H P^-_tH^T+R)^{-1}$$
$$ x_t=x_t^- + K_t(z_t -Hx_t^-)$$
$$P_t=(I-K_t H) P^-_t.$$
%
There are extensions of the Kalman filter (Extended Kalman Filters, or EKF)
for the cases in which the process to be estimated or the measurement-to-process
relation is nonlinear. We do not discuss them here. 
%Basically, % we change our matrixes $A$ and $H$ for some nonlinear functions $f$ and $h$ A Kalman filter that 
%that linearizes about the current mean and covariance.

In our case we consider the input data sequence of real values $z_1, z_2, \ldots,$ $ z_t, \ldots$ 
as the measurement data. The difference equation of our discrete-time controlled process is the simpler one, 
with $A=1, H=1, B=0$. So the equations are simplified to:
%
$$K_t= P_{t-1}/(P_{t-1}+R)$$
$$X_t=X_{t-1}+ K_t(z_t -X_{t-1})$$
$$P_t=P_t(1-K_t)+Q.$$
%

Note the similarity between this Kalman filter and an EWMA estimator, taking $\alpha = K_t$.
This Kalman filter can be considered as an adaptive EWMA estimator where $\alpha = f(Q,R)$ is calculated
optimally when $Q$ and $R$ are known.

The performance of the Kalman filter depends on the accuracy of the a-priori assumptions:
\begin{itemize}
\item linearity of the difference stochastic equation%, and normal probabilities with zero mean of covariances Q and R.
\item estimation of covariances $Q$ and $R$, assumed to be fixed, known, 
      and follow normal distributions with zero mean.
\end{itemize}  
%
When applying the Kalman filter to data streams that vary arbitrarily over time, both
assumptions are problematic. The linearity assumption for sure, but also the assumption
that parameters $Q$ and $R$ are fixed and known -- in fact, estimating them from the data
is itself a complex estimation problem. 



{\tt ADWIN} is basically a linear Estimator with Change Detector that makes an efficient use of 
Memory. It seems a natural idea to improve its performance by replacing the linear estimator by 
an adaptive Kalman filter, where the parameters $Q$ and $R$ of the Kalman filter are computed
using the information in {\tt ADWIN}'s memory. 

We have set $R=W^2/50$ and $Q=200/W$, where $W$ is the length of the window
maintained by {\tt ADWIN}. While we cannot rigorously prove that these are the optimal choices, 
we have informal arguments that these are about the ``right'' forms for $R$ and $Q$, on 
the basis of the theoretical guarantees of {\tt ADWIN}. 

Let us sketch the argument for $Q$. Theorem \ref{ThBV}, part (2) gives a value $\epsilon$
for the maximum change that may have occurred within the window maintained 
by {\tt ADWIN}. This means that the process variance within that window is at most $\epsilon^2$, 
so we want to set $Q=\epsilon^2$. 
In the formula for $\epsilon$, consider the case in which $n_0 = n_1 = W/2$, then we have
$$
\epsilon \ge 4\cdot \sqrt{{\frac{3 (\mu_{W_0}+\epsilon)}{W/2}} \cdot \ln{\frac{4W}{\delta}} }
$$
Isolating from this equation and distinguishing the extreme cases in which 
$\mu_{W_0} \gg \epsilon$ or $\mu_{W_0} \ll \epsilon$, it can be shown that 
$Q=\epsilon^2$ has a form that varies between $c/W$ and $d/W^2$. Here, $c$ and $d$ are constant
for constant values of $\delta$, and $c=200$ is a reasonable estimation. This justifies
our choice of $Q=200/W$. A similar, slightly more involved argument, 
can be made to justify that reasonable values of $R$ are in the range $W^2/c$ to $W^3/d$, 
for somewhat large constants $c$ and $d$.

When there is no change, {\tt ADWIN} window's length increases, 
so $R$ increases too and $K$ decreases, reducing the significance 
of the most recent data arrived. 
Otherwise, if there is change, {\tt ADWIN} window's length reduces, 
so does $R$, and $K$ increases, which means giving more importance to the last data arrived.

%%%%%%%%%%%%%%%%%%%%%%%%%%%%%%%%%%%%%%%%%%%%%%%%%%%%%%%


%% Timememory.tex
%
%%%%%%%%%%%%%%%%%%%%%%%%%%%%%%%%%%%%%%%%%%%%%%%%%%%%%%%

\section{Time and Memory Requirements}

In the experiments above we have only discussed
the performance in terms of error rate, and not 
time or memory usage. Certainly, this was not our main goal
%in this paper, 
and we have in no way tried to optimize 
our implementations in either time or memory (as is clearly indicated
by the choice of Java as programming language). 
Let us, however, mention some rough figures about time and memory, 
since they suggest that our approach can be fairly competitive 
after some optimization work. 

All programs were implemented in Java Standard Edition.
The experiments were performed on a 3.0 GHz Pentium PC machine with 1 Gigabyte main memory,
running Microsoft Windows XP.  The Sun Java 2 Runtime Environment, Standard Edition
(build 1.5.0 06-b05) was used to run all benchmarks.

Consider first the experiments on \adwintwo alone. 
A bucket formed by an integer plus a real number uses~9 bytes. Therefore, 
about 540 bytes store a sliding window of 60 buckets. 
In the boolean case, we could use only 5 bytes per bucket, 
which reduces our memory requirements to 300 bytes per window of 60 buckets.
Note that 60 buckets, with our choice of $M=5$ suffice to represent a window
of length about $2^{60/5} = 4096$. 

%\hrule
%REVISAR I COMPLETAR AQUEST PARAGRAF i resoldre les ref's a les taules
%\hrule
In the experiment comparing different estimators (Tables \ref{tab:compareL11},\ref{tab:compareL13},\ref{tab:compareL21} and \ref{tab:compareL23}), 
the average number of buckets used by \adwintwo was 45,11, and the average time spent was 23 seconds 
to process the $10^6$ samples, which is quite remarkable.
In the Na\"{\i}ve Bayes experiment (Table \ref{tab:NB}), 
it took an average of 1060 seconds and 2000 buckets to process $10^6$ samples
by $34$ estimators. 
%\hrule
%AQUESTS 34 ESTIMATORS SEGUEIX ESSENT CORRECTE? I VOL DIR 34 ESTIMADORS ADWIN, 
%NO COMPTEM ELS FIXED-SIZE WINDOWS NI ELS DE GAMA, oi?
%------ si, �s correcte : s�n 
%------- 8 atributs x 2 valors atribut x 2 clases + 2 clases = 32+2=34
%------- per cada experiment es necesiten 34 estimadors diferents.. 
%\hrule
This means less than $32$ seconds and $60$ buckets per estimator. 
%%Per la versi� llarga nom�s?
The results for $k$-means were similar: We executed the $k$-means
experiments with $k=5$ and two attributes, with 10 estimators and $10^6$ sample points using
about an average of 60 buckets and 11.3 seconds for each instance of \adwintwoz.

\begin{table}[htpb]
%\centering
%\vskip 0.15in
\begin{center}
\begin{tabular}{lccccc}\toprule
Change scale &\adwintwo& Counter& EWMA & Counter& EWMA+Cusum \\
   $n$          &         &        & +Cusum            & Object& Object \\
 \midrule
30	&72,396	&23	&40	&82	&108 \\
50	&72,266	&21	&32	&58	&71 \\
75	&12,079	&17	&23	&54	&66 \\
100	&12,294	&16	&23	&50	&67 \\
1,000	&22,070	&15	&20	&52	&89 \\
10,000	&38,096	&16	&20	&63	&64 \\
100,000	&54,886	&16	&27	&54	&64 \\
1,000,000	&71,882	&15	&20	&59	&64 \\ \bottomrule
\end{tabular}
\end{center}
%\vskip -0.1in
\caption{Time in miliseconds on \adwintwo experiment reading examples from memory}
\label{tab:ADWINM1}
\end{table}

\begin{table}[htpb]
%\centering
%\vskip 0.15in
\begin{center}
\begin{tabular}{lccc}\toprule
Change scale &\adwintwo& Counter& EWMA+  \\
  $n$           &         &        & Cusum   \\
 \midrule
30	&83,769	&10,999	&11,021\\
50	&83,934	&11,004	&10,964\\
75	&23,287	&10,939	&11,002\\
100	&23,709	&11,086	&10,989\\
1,000	&33,303	&11,007	&10,994 \\
10,000	&49,248	&10,930	&10,999 \\
100,000	&66,296	&10,947	&10,923 \\
1,000,000	&83,169	&10,926	&11,037 \\ \bottomrule
\end{tabular}
\end{center}
%\vskip -0.1in
\caption{Time in miliseconds on \adwintwo experiment reading examples from disk}
\label{tab:ADWINM2}
\end{table}

Finally, we compare the time needed by an \adwintwoz, a simple counter, and an EWMA with Cusum change detector and predictor.
We do the following experiment: we feed an \adwintwo estimator, a simple counter and a EWMA with Cusum with $1,000,000$ samples from a distribution
that has an abrupt change every $n$ samples. 
 Table~\ref{tab:ADWINM1} shows the results when the samples are retrieved from memory, and  Table~\ref{tab:ADWINM2} when the samples are stored and retrieved from disk. We test also the overhead due to the fact of using objects, instead of native numbers in Java.
Note that the time difference between \adwintwo and the other methods is not constant, and it depends on the scale of change. The time difference between a EWMA and Cusum estimator and a simple counter estimator is small.
We observe that the simple counter is the fastest method, and that \adwintwo needs more time to process the samples when there is constant change or when there is no change at all.



